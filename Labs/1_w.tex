\section{Lab - Week 1}

\subsection{Supplementary Reading}

\subsubsection{Cell Structure}
Cells are the smallest living units of an organism. They contain the: cell membrane, cytoplasm and DNA. 

There are Eukaryotik cells, which include the nucleus etc, they are more advanced and found in plants or animals. 

There are Prokaryotic cells, which are unicellular, there is genetic information within the cell, however it is not contained by a nucleus.

Organelle are specialised part of the cells:
Nucleus: contains DNA, dictates the operation of the sell. Chromatin is found within the nuclear membrane. When the cell is ready to divide the DNA deconstructs into chromosomes. Nucleus also contains the nucleolus, where ribozomes are made, which synthesise proteins. 

Cytoplasm: Ribozomes wonder freely, or attach to the endoplasmic reticulum (rough and smooth). The ER is a membrane enclosed pathway for transporting materials (proteins). They emerge from the ER in vesicles, where the Golgi body, where they are customised into cells. The Golgi body can add lipids/carbohydrates.

Vacuoulse, they store things... Lysosome, take in damage, which are filled with enzymes for processing dead cells etc.

The mitochondrion are the powerhouse of the cell. They make ATP molecules as part of cellular respiration. 

Cytoskeleton, the thread like microfilaments and microtubules are made of protein. 

Plants contains chloroplast, where photosynthesis occurs.

Unique structures: In the human lung, there are cilia which aim to collect debris. Bacteria contain a flagella, which helps a cell propel itself. Flagella is also present in sperm cells. 

Summary
Eukaryotic Cells: Plant and animal cells with a nucleus and membrane-enclosed organelles
Prokaryotic Cells: Unicellular organisms without a nucleus or membrane enclosed organelles
All Cells: Have a cell membrane, cytoplasm and genetic material
Plant and Animal Cells: Contain mitochondria.

\subsubsection{From DNA to Protein}

The nucleus contains the genome, which is split between 23 pairs of chromosomes. Each chromosome is packaged around proteins called histones. Within DNA are sections called genes, which contain instructions for making proteins. 

When a gene is switched on, an enzyme called RNA polymerase attaches to the start of the gene. It moves along the DNA making a stream of mRNA from the free bases. The DNA code contains the order in which the free bases are added. This is called transcription. Before mRNA is a template, we must remove and add sections of RNA. mRNA then moves into the cytoplasm, and protein factories (ribosomes) reads the code to produce an amino acid chain. tRNA modules carry the amino acids to the ribosome. This is read in groups of three, and added to a growing chain of amino acids. The protein is then formed as a complex shape.

\subsubsection{Interphase Changes}

\textbf{$G_1$ Phase} The first growth phase. This is at the start of DNA synthesis, biosynthetic activities are at a high rate. The $G_1$ duration is highly variable, even among different cells of the same species. In this phase, the cell increases its supply of proteins, organelles such as mitochondria and mitochondria, and grows in size. It continues to the S phase, although it can go to $G_0$ or re enter the cell cycle.
\\

\noindent\textbf{$S$ Phase} DNA is replicated in this phase. When it is complete, all of the chromosomes have been replicated, leaving two sister chromatids. The amount of DNA in the cell has doubled, though the number of ploidy and chromosomes are unchanged. RNA transcription and protein synthesis are very low during this phase.
\\

\noindent\textbf{$G_2$ Phase} is a period of protein synthesis and rapid cell growth in preparation for mitosis. During this phase microtubules begin to reorganize to form a spindle. Cells are checked in this phase to check for DNA damage. This is regulated by the tumor protein $p53$, which will repair the DNA or trigger the apoptosis of the cell, as mentioned previously lyosomes, are a mediator for apoptosis.




