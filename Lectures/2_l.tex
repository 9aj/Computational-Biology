\section{Cell Regulation}

\subsection{Cells}
Cells are complicated machines, understanding how they work is complicated. \\

\noindent Prokaryotes and Eukaryotes.
\begin{itemize}
    \itemsep0em
    \item Membrane-enclosed nucleus
    \item Nucleolus
    \item Mitochondria
\end{itemize}

\begin{itemize}
    \itemsep0em
    \item Share cell membrane, DNA code, ribosome (protein machine)
    \item Eukaryotes are much more complex
    \item It is now generally accepted that Eukaryotes were formed by fusing different cells.
    \begin{itemize}
        \itemsep0em
        \item nucleus - would have been membrane of the original cell.
        \item mitochrondria - create ATP for energy. Originally a bacteria that became part of the cell. They have their own DNA, which is only inherited from female chromosomes.
    \end{itemize}
    \item Eukaryotes can be multicellular, yeast is an example of an Eukaryotes.
\end{itemize}

\subsection{Proteins}
\begin{itemize}
    \itemsep0em
    \item Proteins are hard to understand, they are sequenced directly. This is difficult due to wrapping.
    \item Fortunately we can deduce their sequence from DNA or mRNA.
    \item Measuring protein concentration is difficult
    \begin{itemize}
        \item Field of proteomics looks at this
        \item Uses tools like mass spectrometer to count proteins
        \item This is becoming more common.
    \end{itemize}
\end{itemize}

\subsubsection{Protein Folding}
We need to determine their three dimensional shape, in some instances we can crystallise them and use X-ray diffraction. Predicting the structure is a non-local problem. One solution, that is particularly recent, is \textit{DeepMinds} alpha-folds.

\footnote{Protein structure prediction using multiple deep neural
networks in the 13th Critical Assessment of Protein
Structure Prediction (CASP13)}

\subsubsection{Secondary Structure}
The predominant components making up proteins are:
\begin{itemize}
    \itemsep0em
    \item Alpha helices - \textbf{they are held by electrostatic bonds}
    \item Beta sheets
    \item Various folds and connecting parts
\end{itemize}

\subsubsection{The First Law of Protein Sequences}

\begin{quote}
    \textit{Similar structures perform similar functions}
\end{quote}

We can learn about the function of a protein by comparing with known proteins. Because of evolution, many proteins are common to a huge number of organisms. However, dissimilar structures can have a similar function (convergent evolution). The structure of a protein is much more stable than the structure of an amino acid.

\subsubsection{Finding Aligned Sequences}
Biologists routinely compare any new sequence with sequences already found. This requires comparisons with very large databases.

\subsection{Cell Regulation}
Cells must respond to their environment, maintain their environment, divide. As of this, they need a control mechanism. Proteins and DNA lie at the heart of cell regulation.

\subsubsection{Promoter Region}
Promoters are important elements for gene expression. The promoter regulates where, when and to what level a gene is expressed. RNA polymerase binds to the promoter region on the gene, leading to the production of mRNA.

\subsubsection{Regulation by Signal}
Coded by the promoter region, transcribed to mRNA then is translated to a protein. This happens at different places on the DNA. Regulation works with signal molecules, they move quickly through the cell to bind to the proteins. They slightly change the shape of the protein, so that the protein can bind to the regulatory region. The protein can then excite the transcription, or inhibition.

Many different proteins can bind to the promoter region

\subsubsection{TATA box}
Close to the coding regions is a region where there are more 

\subsubsection{Gene Regulation Networks}
\begin{itemize}
    \itemsep0em
    \item Proteins can regulate their own expression. 
    \item Proteins can regulate many other proteins
    \item Protein $A$ can regulate protein $B$, which in turn regulates protein $A$ (if these are inhibitory then you get a flip flop). This forms a complex circuit.
\end{itemize}

GRN are dynamic, we need to simulate the network. We need to know the rate at which the molecules are produced.

\subsubsection{Motifs}
To understand what is going on we can look for patterns or 'motifs'. These are things which happen more than by chance. Synthetic biology actually builds biological circuits from these motifs.

\subsubsection{Multicellular}
Many Eukaryotes are multicellular, in many cases the cells differentiate (however in a jellyfish, all the cells are identical). All the cells have the same DNA, but are reading different parts of the DNA.

\subsection{Summary}
We are the beginning of understanding how cells work. Much of our understanding comes from comparing different organisms. We are beginning to model some of the basic interactions that make cells function. As of this, we need a host of computation tools.



