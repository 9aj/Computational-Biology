\section{Lecture 1: Introduction}

\subsection{Big Data}
Biology is a big data problem
\begin{itemize}
    \itemsep0em
    \item Genomes involve $10^5$ - $10^{12}$ nucleotide bases
    \item Number of different proteins can be $10^5$
    \item Number of different cell types $10^2$
    \item Number of species $10^7$
\end{itemize}

As a result of this, biology involves a lot of information, of which is stored in big databases.

\par\noindent Tools of computer science are essential to store the data, and make inferences regarding the data. The tools of abstraction are equally valid in understanding biology as they are in computer science.

\subsection{Interactions}

Interactions occur across many levels
\begin{itemize}
    \itemsep0em
    \item Protein-protein interaction are common place. Form complex's and work together
    \item Protein-DNA interaction underlie cell regularisation
    \item Protein-environment interaction are how the cells sense the world
    \item Species interact in ecosystems. We are in an ecosystem! Half of the cells in our body aren't ours (Ecoli)...
\end{itemize}

If $P\approx 10^5$ is the number of proteins, then the number of possible interacting pairs are $P\approx 10^{10}$

\subsection{Biology Self-Organises}
Proteins complexes form machines, life organises into cells, cells can organise into higher organisms and these organisms form populations that act together.

\subsection{Life is Dynamic}
Cells are chemical machines, they work through complex auto-regulation, where cells divide, and fuse sexually. This resolves to evolution.

\subsection{Evolution}

\begin{quote}
    "Nothing in biology makes sense except in the light of evolution" - Theodosius Dobzhansky
\end{quote}
At every level, from protein to species, we can make sense of this in terms of evolution.

\subsection{The Tree of Life}
All observations are consistent with life on earth being related through a tree like structure.

\par\noindent At the molecular level, different species share a huge amount in common. We use this commonality to transfer knowledge between species.

\subsection{Inheritance}
When Darwin wrote the origin of species, inheritance was not generally understood. This caused confusion because \textbf{blending inheritance} would seem to remove the variation needed for natural selection.

\par\noindent Modern inheritance theory was initially proposed by Gregor Mendel in 1860, but was not widely known.

\par\noindent Around 1900, it was finally widely understood that inheritance occurred through the exchange of discrete units.

\subsection{Gregor Mendel's Peas}
Mendel took different kinds of garden peas. His explanation was that the experiment, every pea carried two copies of genes, one from each parent. He would pollinate flowers and allow the offspring the self pollinate. This would show the diversity in the genes of the self pollinating parent.