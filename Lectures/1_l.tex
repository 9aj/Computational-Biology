\section{Lecture 1: Introduction}

\subsection{Big Data}
Biology is a big data problem
\begin{itemize}
    \itemsep0em
    \item Genomes involve $10^5$ - $10^{12}$ nucleotide bases
    \item Number of different proteins can be $10^5$
    \item Number of different cell types $10^2$
    \item Number of species $10^7$
\end{itemize}

As a result of this, biology involves a lot of information, of which is stored in big databases.

\par\noindent Tools of computer science are essential to store the data, and make inferences regarding the data. The tools of abstraction are equally valid in understanding biology as they are in computer science.

\subsection{Interactions}

Interactions occur across many levels
\begin{itemize}
    \itemsep0em
    \item Protein-protein interaction are common place. Form complex's and work together
    \item Protein-DNA interaction underlie cell regularisation
    \item Protein-environment interaction are how the cells sense the world
    \item Species interact in ecosystems. We are in an ecosystem! Half of the cells in our body aren't ours (Ecoli)...
\end{itemize}

If $P\approx 10^5$ is the number of proteins, then the number of possible interacting pairs are $P\approx 10^{10}$

\subsection{Biology Self-Organises}
Proteins complexes form machines, life organises into cells, cells can organise into higher organisms and these organisms form populations that act together.

\subsection{Life is Dynamic}
Cells are chemical machines, they work through complex auto-regulation, where cells divide, and fuse sexually. This resolves to evolution.

\subsection{Evolution}

\begin{quote}
    "Nothing in biology makes sense except in the light of evolution" - Theodosius Dobzhansky
\end{quote}
At every level, from protein to species, we can make sense of this in terms of evolution.

\subsection{The Tree of Life}
All observations are consistent with life on earth being related through a tree like structure.

\par\noindent At the molecular level, different species share a huge amount in common. We use this commonality to transfer knowledge between species.

\subsection{Inheritance}
When Darwin wrote the origin of species, inheritance was not generally understood. This caused confusion because \textbf{blending inheritance} would seem to remove the variation needed for natural selection.

\par\noindent Modern inheritance theory was initially proposed by Gregor Mendel in 1860, but was not widely known. \\

\par\noindent Around 1900, it was finally widely understood that inheritance occurred through the exchange of discrete units.

\subsection{Gregor Mendel's Peas}
Mendel took different kinds of garden peas. His explanation was that the experiment, every pea carried two copies of genes, one from each parent. He would pollinate flowers and allow the offspring the self pollinate. This would show the diversity in the genes of the self pollinating parent.

\subsubsection{Inheritance}
Mendels experiments could be explained if:
\begin{itemize}
    \itemsep0em
    \item Traits were carried by discrete units (genes)
    \item Each individual carries two copies of a gene (diploidy), bees are (haploids), plants are (multiplois).
    \item We inherit one copy from each parent
    \item One gene is dominant
\end{itemize}

\noident These are not always true. The experiments were in fact rigged. The mechanism for inheritance was not understood until we understood DNA.

\subsection{DNA}
Crick and Watson found the double helix. We have base pairs, and the double helix is held together by 4 types of bases. Each base, A, C, G, T, form base pairs A-T or C-G.

\begin{itemize}
    \itemsep0em
    \item A - Adenine
    \item C - Cytosine
    \item G - Guanine
    \item T - Thymine
\end{itemize}

\subsubsection{DNA Replication}
Chromosones have long strands of folded DNA. We have an original DNA template, which forms a replication fork. We have free nucleotides, a leading strand and the original template strand.

\subsection{What does DNA do?}
Parts of DNA codes are for Proteins. We copy to messenger RNA through transcription, mRNA then gets translated to proteins. \\

\par\noindent A lot of DNA does not code for protein, can code for RNA molecules (ribozymes), can be used to control cells, or can be junk DNA from replication.

\subsubsection{RNA and DNA}
RNA (Ribonucleic Acid) forms a single helix, it has a different base, Uracil, instead of Thymine. The backbone of RNA is not rigid, meaning it can flex and turn on itself.

\subsubsection{RNA Polymerase}
DNA is transcribed to RNA. DNA is unwinded. DNA is copied to RNA with an mRNA transcript. RNA is rewinded to DNA.

\subsubection{Exons and Introns}
Introns are ignored. Exons are transcripted, by eliminating the intron transcript segments, and the splicing of exons.

\subsubection{mRNA to Protein Translation}

\textbf{Likely non-examinable} Growing polypeptide chains, mRNA is read in triplets (every three bases), and they are matched to complement bases tRNA.
\subsection{}